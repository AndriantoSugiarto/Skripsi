\documentclass[a4paper,twoside]{article}
\usepackage[T1]{fontenc}
\usepackage[bahasa]{babel}
\usepackage{graphicx}
\usepackage{graphics}
\usepackage{float}
\usepackage[cm]{fullpage}
\pagestyle{myheadings}
\usepackage{etoolbox}
\usepackage{setspace} 
\usepackage{lipsum} 
\setlength{\headsep}{30pt}
\usepackage[inner=2cm,outer=2.5cm,top=2.5cm,bottom=2cm]{geometry} %margin
% \pagestyle{empty}

\makeatletter
\renewcommand{\@maketitle} {\begin{center} {\LARGE \textbf{ \textsc{\@title}} \par} \bigskip {\large \textbf{\textsc{\@author}} }\end{center} }
\renewcommand{\thispagestyle}[1]{}
\markright{\textbf{\textsc{AIF401/AIF402 \textemdash Rencana Kerja Skripsi \textemdash Sem. Genap 2017/2018}}}

\onehalfspacing
 
\begin{document}

\title{\@judultopik}
\author{\nama \textendash \@npm} 

%tulis nama dan NPM anda di sini:
\newcommand{\nama}{Andrianto Sugiarto}
\newcommand{\@npm}{2013730046}
\newcommand{\@judultopik}{Migrasi SIAModels dan IFStudentPortal ke Kurikulum 2018} % Judul/topik anda
\newcommand{\jumpemb}{1} % Jumlah pembimbing, 1 atau 2
\newcommand{\tanggal}{26/01/2018}
\maketitle

\pagenumbering{arabic}

\section{Deskripsi}

\paragraph{} IFStudentPortal merupakan sistem informasi berbasis \textit{web} untuk mahasiswa Teknik Informatika UNPAR. IFStudentPortal sendiri adalah aplikasi buatan Herfan Heryandi dan kontributor lainnya. Fitur-fitur yang dimiliki IFStudentPortal yaitu persiapan perwalian, jadwal kuliah, dan syarat kelulusan. Saat ini IFStudentPortal dan SIAModels mendukung kurikulum 2013. SIAModels sendiri merupakan kelas-kelas dalam bahasa java yang merepresentasikan Sistem Informasi Akademik Teknik Informatika UNPAR.

Pada skripsi ini, akan ditambahkan fitur untuk mengonversi kurikulum 2013 ke kurikulum 2018 terutama untuk mahasiswa yang sudah mengambil kuliah-kuliah pada kurikulum 2013. Selain itu, IFStudentPortal dan SIAModels yang tadinya mendukung kurikulum 2013 akan diubah ke kurikulum 2018 yang terlihat ada perbedaan dari kurikulum sebelumnya.



\section{Rumusan Masalah}
Rumusan dari masalaah yang akan dibahas pada skripsi ini sebagai berikut:
\begin{itemize}
	\item Bagaimana mengonversi SIAModels dan IFStudentPortal, sehingga mendukung kurikulum 2018 serta konversinya (untuk mahasiswa yang sudah mengambil kuliah-kuliah di kurikulum 2013)?
	\item Bagaimana mengimplementasikan kurikulum 2018 untuk SIAModels dan IFStudentPortal?
\end{itemize}

\section{Tujuan}
Tujuan yang ingin dicapai dalam penulisan skripsi ini sebagai berikut:
\begin{itemize}
	\item Mengonversi SIAModels dan IFStudentPortal untuk mendukung kurikulum 2018 dan konversi dari mata kuliah kurikulum 2013 ke 2018.
	\item Mengimplementasikan kurikulum 2018 untuk SIAModels dan IFStudentPortal.
\end{itemize}

\section{Deskripsi Perangkat Lunak}
Perangkat lunak akhir yang akan dibuat memiliki fitur minimal sebagai berikut:
\begin{itemize}
	\item Perangkat lunak mendukung kurikulum 2018
	\item Perangkat lunak mampu mengonversi kuliah-kuliah kurikulum 2013 ke kurikulum 2018
\end{itemize}

\section{Detail Pengerjaan Skripsi}
Bagian-bagian pekerjaan skripsi ini adalah sebagai berikut :
	\begin{enumerate}
		\item Mempelajari IFStudentPortal dan SIAModels saat ini
		\item Melakukan Studi Literatur mengenai Kurikulum 2018, \textit{library} jsoup  dan Play Framework
		\item Melakukan konversi dari mata kuliah kurikulum 2013 ke kurikulum 2018 
		\item Mengimplementasikan kurikulum 2018 untuk IFStudentPortal dan SIAModels
		\item Melakukan pengujian dan eksperimen
		\item Menulis dokumen skripsi
	\end{enumerate}

\section{Rencana Kerja}
Rencana kerja dibagi menjadi dua bagian yaitu yang akan dilakukan pada saat mengambil kuliah AIF401 Skripsi 1 dan pada saat mengambil kuliah AIF402 Skripsi 2. diuraikan sebagai berikut :


\begin{center}
  \begin{tabular}{ | c | c | c | c | l |}
    \hline
    1*  & 2*(\%) & 3*(\%) & 4*(\%) &5*\\ \hline \hline
    1   & 10  & 10  &  &  \\ \hline
    2   & 10 & 10  &   & \\ \hline
    3   & 10  &   & 10 & \\ \hline
    4   & 35  & 15  & 20 &  \\ \hline
    5   & 20  &   & 20 & \\ \hline
    6   & 15  & 7  & 8  & {\footnotesize sebagian bab 1,2 dan 3 dikerjakan di skripsi 1}\\ \hline
    Total  & 100  & 42  & 58 &  \\ \hline
                          \end{tabular}
\end{center}

Keterangan (*)\\
1 : Bagian pengerjaan Skripsi (nomor disesuaikan dengan detail pengerjaan di bagian 5)\\
2 : Persentase total \\
3 : Persentase yang akan diselesaikan di Skripsi 1 \\
4 : Persentase yang akan diselesaikan di Skripsi 2 \\
5 : Penjelasan singkat apa yang dilakukan di S1 (Skripsi 1) atau S2 (skripsi 2)

\vspace{1cm}
\centering Bandung, \tanggal\\
\vspace{2cm} \nama \\ 
\vspace{1cm}

Menyetujui, \\
\ifdefstring{\jumpemb}{2}{
\vspace{1.5cm}
\begin{centering} Menyetujui,\\ \end{centering} \vspace{0.75cm}
\begin{minipage}[b]{0.45\linewidth}
% \centering Bandung, \makebox[0.5cm]{\hrulefill}/\makebox[0.5cm]{\hrulefill}/2013 \\
\vspace{2cm} Nama: \makebox[3cm]{\hrulefill}\\ Pembimbing Utama
\end{minipage} \hspace{0.5cm}
\begin{minipage}[b]{0.45\linewidth}
% \centering Bandung, \makebox[0.5cm]{\hrulefill}/\makebox[0.5cm]{\hrulefill}/2013\\
\vspace{2cm} Nama: \makebox[3cm]{\hrulefill}\\ Pembimbing Pendamping
\end{minipage}
\vspace{0.5cm}
}{
% \centering Bandung, \makebox[0.5cm]{\hrulefill}/\makebox[0.5cm]{\hrulefill}/2013\\
\vspace{2cm} Nama: \makebox[3cm]{\hrulefill}\\ Pembimbing Tunggal
}

\end{document}

