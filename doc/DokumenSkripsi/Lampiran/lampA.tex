%versi 3 (18-12-2016)
\chapter{Kode Program \textit{Controller}}
\label{lamp:A}

\singlespacing
\begin{lstlisting}[language=Java,basicstyle=\tiny,caption=Applicstion.java]
package controllers;

import java.io.IOException;
import java.util.*;

import models.display.DataAkademikDisplay;
import models.display.JadwalDisplay;
import models.display.PrasyaratDisplay;
import models.display.KelulusanDisplay;
import id.ac.unpar.siamodels.JadwalKuliah;
import id.ac.unpar.siamodels.Mahasiswa;
import id.ac.unpar.siamodels.Mahasiswa.Nilai;
import id.ac.unpar.siamodels.MataKuliah;
import id.ac.unpar.siamodels.Semester;
import id.ac.unpar.siamodels.TahunSemester;
import id.ac.unpar.siamodels.matakuliah.interfaces.HasPrasyarat;
import id.ac.unpar.siamodels.prodi.teknikinformatika.*;
import models.support.Scraper;
import play.data.DynamicForm;
import play.data.Form;
import play.mvc.*;
import play.Logger;

import javax.script.ScriptException;

public class Application extends Controller {

	public Result index() throws IOException {
		if (session("npm") == null) {
			return ok(views.html.login.render(""));
		} else {
			return home();
		}
	}

	public Result login() throws IOException {
		if (session("npm") == null) {
			return index();
		} else {
			return home();
		}
	}

	public Result submitLogin() throws IOException {
		Scraper scrap = new Scraper();
		String errorHtml = "<div class='alert alert-danger' role='alert'>"
				+ "<span class='glyphicon glyphicon-exclamation-sign' aria-hidden='true'></span>"
				+ "<span class='sr-only'>Error:</span>";
		DynamicForm dynamicForm = Form.form().bindFromRequest();
		String email = dynamicForm.get("email");
		String pass = dynamicForm.get("pass");
		if (!email.matches("[0-9]{7}+@student.unpar.ac.id") && !email.matches("[0-9]{10}+@student.unpar.ac.id")) {
			Logger.info(
					"User: " + email + " gagal login dari " + request().remoteAddress() + " karena e-mail tidak valid");
			return ok(views.html.login.render(errorHtml + "Email tidak valid" + "</div>"));
		}
		if (!(email.charAt(0) == '7' && email.charAt(1) == '3' || email.contains("201773"))) {
			Logger.info("User: " + email + " gagal login dari " + request().remoteAddress()
					+ " karena bukan mahasiswa teknik informatika (73*)");
			return ok(views.html.login.render(errorHtml + " bukan mahasiswa teknik informatika" + "</div>"));
		}
		String npm = "";
		if(email.contains("201773")){
		    npm = email.substring(0,10);
        } else {
            npm = "20" + email.substring(2, 4) + email.substring(0, 2) + "0" + email.substring(4, 7);
        }
		String phpsessid = scrap.login(npm, pass);
		if (phpsessid != null) {
			Logger.info("User " + email + " berhasil login dari " + request().remoteAddress());
			session("npm", npm);
			session("email", email);
			session("phpsessid", phpsessid);
			return home();
		} else {
			Logger.info("User: " + email + " gagal login dari " + request().remoteAddress()
					+ " karena input password salah atau bukan mahasiswa aktif");
			return ok(views.html.login
					.render(errorHtml + "Password yang anda masukkan salah atau bukan mahasiswa aktif" + "</div>"));
		}
	}

	public Result home() throws IOException {
		if (session("npm") == null || session("phpsessid") == null) {
			session().clear();
			return index();
		} else {
			Scraper scrap = new Scraper();
			Mahasiswa mhs = new Mahasiswa(session("npm"));
			scrap.requestNamePhotoTahunSemester(session("phpsessid"), mhs);
			Logger.info("User " + session("email") + " mengakses halaman home dari " + request().remoteAddress());
			return ok(views.html.home.render(mhs));
		}
	}

	public Result perwalian() throws IOException, InterruptedException {

		if (session("npm") == null || session("phpsessid") == null) {
			session().clear();
			return index();
		} else {
			String phpsessid = session("phpsessid");
			Logger.info("User " + session("email") + " mengakses halaman prasyarat dari " + request().remoteAddress());
			Mahasiswa mhs = new Mahasiswa(session("npm"));
			Scraper scrap = new Scraper();
			TahunSemester currTahunSemester = scrap.requestNamePhotoTahunSemester(session("phpsessid"), mhs);
			scrap.requestAvailableKuliah(phpsessid);
			scrap.requestNilaiTOEFL(phpsessid, mhs);
			//List<JadwalKuliah> jadwalList = scrap.requestJadwal(phpsessid);
			//mhs.setJadwalKuliahList(jadwalList);
			scrap.requestNilai(phpsessid, mhs);
			DataAkademikDisplay dataAkademik = new DataAkademikDisplay();
			List<Nilai> riwayatNilai = mhs.getRiwayatNilai();

			if (riwayatNilai.size() == 0) {
				List<PrasyaratDisplay> table = null;
				String currentSemester = currTahunSemester.getSemester() + " " + currTahunSemester.getTahun() + "/"
						+ (currTahunSemester.getTahun() + 1);
				return ok(views.html.perwalian.render(table, currentSemester, dataAkademik));
			} else {
				dataAkademik.ips = String.format("%.2f", mhs.calculateIPS());
				dataAkademik.ipKumulatif = String.format("%.2f", mhs.calculateIPKumulatif());
				dataAkademik.ipLulus = String.format("%.2f", mhs.calculateIPLulus());
				dataAkademik.ipNTerbaik = String.format("%.2f", mhs.calculateIPTempuh(false));
				dataAkademik.sksLulusTotal = mhs.calculateSKSLulus();
				dataAkademik.nilaiTOEFL = "" + mhs.getNilaiTOEFL().values();

				int lastIndex = riwayatNilai.size() - 1;
				Semester semester = riwayatNilai.get(lastIndex).getSemester();
				int tahunAjaran = riwayatNilai.get(lastIndex).getTahunAjaran();
				int totalSKS = 0;
				for (int i = lastIndex; i >= 0; i--) {
					Nilai nilai = riwayatNilai.get(i);
					if (nilai.getSemester() == semester && nilai.getTahunAjaran() == tahunAjaran) {
						if (nilai.getAngkaAkhir() != null) {
							totalSKS += nilai.getMataKuliah().getSks();
						}
					} else {
						break;
					}
				}
				String semTerakhir = semester + " " + tahunAjaran + "/" + (tahunAjaran + 1);
				dataAkademik.semesterTerakhir = semTerakhir;
				dataAkademik.sksLulusSemTerakhir = totalSKS;

				List<PrasyaratDisplay> table = checkPrasyarat();
				String currentSemester = currTahunSemester.getSemester() + " " + currTahunSemester.getTahun() + "/"
						+ (currTahunSemester.getTahun() + 1);
				return ok(views.html.perwalian.render(table, currentSemester, dataAkademik));
			}
		}
	}

	public Result jadwalKuliah() throws IOException {

		if (session("npm") == null || session("phpsessid") == null) {
			session().clear();
			return index();
		} else {
			Scraper scrap = new Scraper();
			Mahasiswa mhs = new Mahasiswa(session("npm"));
			List<JadwalKuliah> jadwalList = scrap.requestJadwal(session("phpsessid"));
			mhs.setJadwalKuliahList(jadwalList);
			TahunSemester currTahunSemester = scrap.requestNamePhotoTahunSemester(session("phpsessid"), mhs);
			JadwalDisplay table = new JadwalDisplay(mhs.getJadwalKuliahList());
			String semester = currTahunSemester.getSemester() + " " + currTahunSemester.getTahun() + "/"
					+ (currTahunSemester.getTahun() + 1);
			Logger.info(
					"User " + session("email") + " mengakses halaman jadwal kuliah dari " + request().remoteAddress());
			return ok(views.html.jadwalKuliah.render(table, semester));
		}
	}

	public Result kelulusan() throws IOException, InterruptedException {
		if (session("npm") == null || session("phpsessid") == null) {
			session().clear();
			return index();
		} else {
			String phpsessid = session("phpsessid"); 
			Logger.info(
					"User " + session("email") + " mengakses halaman Data akademik dari " + request().remoteAddress());
			Scraper scrap = new Scraper();
			Mahasiswa mhs = new Mahasiswa(session("npm"));
			scrap.requestNilai(phpsessid, mhs);
			scrap.requestNilaiTOEFL(phpsessid, mhs);
			if (mhs.getRiwayatNilai().size() == 0) {
				KelulusanDisplay display = null;
				return ok(views.html.kelulusan.render(display));
			} else {
				Mahasiswa currMahasiswa = mhs;
				KelulusanDisplay display = new KelulusanDisplay();
				Kelulusan str = new Kelulusan();
				ArrayList<String> arrString = new ArrayList<>();
				str.checkPrasyarat(currMahasiswa, arrString);
				display.alasanBelumLulus = arrString;
				return ok(views.html.kelulusan.render(display));
			}
		}
	}

	public Result tentang() throws IOException {
		if (session("npm") == null || session("phpsessid") == null) {
			session().clear();
			return index();
		} else {
			Logger.info("User " + session("email") + " mengakses halaman info dari " + request().remoteAddress());
			return ok(views.html.tentang.render());
		}
	}

	public Result logout() throws IOException {
		Logger.info("User " + session("email") + " telah logout dari " + request().remoteAddress());
		session().clear();
		return index();
	}

	private List<PrasyaratDisplay> checkPrasyarat() throws IOException, InterruptedException {
		Scraper scrap = new Scraper();
		Mahasiswa mhs = new Mahasiswa(session("npm"));
		scrap.requestNilai(session("phpsessid"), mhs);
		List<PrasyaratDisplay> table = new ArrayList<PrasyaratDisplay>();
		List<MataKuliah> mkList = scrap.requestAvailableKuliah(session("phpsessid"));
		for (MataKuliah mk : mkList) {
			if (mhs.hasLulusKuliah(mk.getKode())) {
				table.add(new PrasyaratDisplay(mk, new String[] { "sudah lulus" }));
			} else {
				if ((Object)mk instanceof HasPrasyarat) {
					List<String> reasons = new ArrayList<String>();
					((HasPrasyarat) mk).checkPrasyarat(mhs, reasons);
					if (!reasons.isEmpty()) {
						table.add(new PrasyaratDisplay(mk, reasons.toArray(new String[reasons.size()])));
					} else {
						if (mhs.hasLulusKuliah(mk.getKode())) {
							table.add(new PrasyaratDisplay(mk, new String[] { "sudah lulus" }));
						} else {
							table.add(new PrasyaratDisplay(mk, new String[] { "memenuhi syarat" }));
						}
					}
				} else {
					table.add(new PrasyaratDisplay(mk, new String[] { "tidak memiliki prasyarat" }));
				}

			}
		}
		return table;
	}
}
\end{lstlisting}
