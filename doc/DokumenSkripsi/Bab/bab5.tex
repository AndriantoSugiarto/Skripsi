\chapter{Implementasi dan Pengujian}
\label{chap:implementasiPengujian}

Bab ini terdiri atas dua bagian, yaitu Implementasi Perangkat Lunak dan Pengujian Perangkat Lunak. Bagian implementasi berisi penjelasan lingkungkan pengembangan perangkat lunak dan hasil implementasi. Sedangkan bagian pengujian berisi hasil pengujian fungsional dan eksperimental terhadap perangkat lunak yang telah dibangun.

\section{Implementasi}
\label{sec:implementasi}

\subsection{Lingkungan Implementasi}
		\label{sec:lingkungan_implementasi}
			Lingkungan implemementasi yang digunakan adalah Heroku. Heroku merupakan \textit{cloud platform} yang menyediakan fitur yang dapat membantu pengguna untuk dapat memiliki alamat domain. Spesifikasi Heroku yang digunakan oleh IFStudentPortal akan dijelaskan sebagai berikut:
			
			\begin{tabular}{ |p{1.5cm}|p{1.5cm}|p{2cm}|p{1.5cm}|p{1.5cm}|p{1.5cm}|p{1.5cm}|}
			\hline
			Dyno Type & Sleeps & Professional Features & Memory (RAM) & CPU Share & Dedicated & Compute \\ \hline
			free & yes & no & 512 MB & 1x & no & 1x-4x \\ \hline
				
			\end{tabular}
				
\subsection{Hasil Implementasi}
		Hasil implementasi berupa aplikasi berbasis web yang dikembangkan untuk menyesuaikan dengan StudentPortal Baru dan Kurikulum 2018. Aplikasi dapat diakses dengan URL \url{https://ifstudentportalskripsi.herokuapp.com}. Aplikasi Informatika Student Portal terdiri dari lima halaman antara lain:
		\begin{enumerate}
			\item\textbf{Halaman \textit{Login}}\\
				Halaman \textit{login} digunakan pengguna untuk masuk ke dalam aplikasi. Pada halaman ini, pengguna dapat melakukan \textit{login} dengan mengisi \textit{email} pada kolom \textit{email} dan \textit{password} pada kolom \textit{password} kemudian mengklik tombol login. Tangkapan layar dari halaman \textit{login} dapat dilihat pada Gambar \ref{fig:5_halaman_login}.
					\begin{figure}[H]
						\centering
						\includegraphics[scale=0.34]{Gambar/halaman_login}
						\caption{Halaman \textit{Login}} 
						\label{fig:5_halaman_login}
					\end{figure}
					
				\item\textbf{Halaman \textit{Home}}\\
				Halaman utama merupakan halaman yang pertama kali dituju setelah melakukan \textit{login}. Halaman utama menampilkan identitas pengguna dan \textit{link} menuju kode sumber aplikasi Informatika Student Portal. Tangkapan layar dari halaman utama dapat dilihat pada Gambar \ref{fig:5_halaman_utama}.
					\begin{figure}[H]
						\centering
						\includegraphics[scale=0.34]{Gambar/halaman_home}
						\caption{Halaman \textit{Home}} 
						\label{fig:5_halaman_utama}
					\end{figure}
						
				\item\textbf{Halaman Persiapan Perwalian}\\
				Halaman ini menampilkan data akademik dan tabel prasyarat mata kuliah. Pengguna dapat mengklik kode mata kuliah, kemudian akan diarahkan ke kode sumber aturan prasyarat mata kuliah tersebut. Tangkapan layar dari halaman prasyarat mata kuliah dapat dilihat pada Gambar \ref{fig:5_halaman_persiapan_perwalian}.
					\begin{figure}[H]
						\centering
						\includegraphics[scale=0.34]{Gambar/halaman_persiapan_perwalian}
						\caption{Halaman Persiapan Perwalian} 
						\label{fig:5_halaman_persiapan_perwalian}
					\end{figure}

				\item\textbf{Halaman Jadwal Kuliah}\\
				Halaman ini menampilkan jadwal kuliah yang tersusun dan terurut berdasarkan hari. Tangkapan layar dari halaman jadwal kuliah dapat dilihat pada Gambar \ref{fig:5_halaman_jadwal}. Jika kode mata kuliah diklik, akan muncul \textit{popup} seperti pada Gambar \ref{fig:5_halaman_jadwal_rinci} yang berisi rincian dari jadwal kuliah tersebut.
				\begin{figure}[H]
						\centering
						\includegraphics[scale=0.34]{Gambar/halaman_jadwal}
						\caption{Halaman Jadwal Kuliah} 
						\label{fig:5_halaman_jadwal}
					\end{figure}
					
					\begin{figure}[H]
						\centering
						\includegraphics[scale=0.34]{Gambar/halaman_jadwal_rinci}
						\caption{Rincian Jadwal Kuliah} 
						\label{fig:5_halaman_jadwal_rinci}
					\end{figure}
					
				\item\textbf{Halaman Syarat Kelulusan}\\
				Halaman ini menampilkan syarat kelulusan dari Program Studi Teknik Informatika yang belum dipenuhi oleh mahasiswa. Tangkapan layar dari halaman data akademik dapat dilihat pada Gambar \ref{fig:5_halaman_syarat_kelulusan}.
				\begin{figure}[H]
						\centering
						\includegraphics[scale=0.34]{Gambar/halaman_syarat_kelulusan}
						\caption{Halaman Syarat Kelulusan} 
						\label{fig:5_halaman_syarat_kelulusan}
					\end{figure}
		\end{enumerate}
		
\section{Pengujian}

\subsection{Pengujian Fungsional}
\label{subsec:fungsional}

Pengujian fungsional dilakukan untuk mengetahui kesesuaian reaksi perangkat lunak dengan reaksi yang diharapkan berdasarkan aksi pengguna terhadap perangkat lunak. Pengujian dilakukan pada \textit{cloud platform} dan Windows dengan hasil yang sama.

			\begin{table}[H]
			\centering
			\caption{Tabel Pengujian Fungsional}
				\begin{tabular}{|p{0.25cm}| p{3.5cm}| p{7cm}| p{2.5cm}|} \hline
				No.	&	Aksi Pengguna	&	Reaksi yang diharapkan	&	Reaksi Perangkat Lunak \\ \hline
				1.	&	Pengguna menjalankan aplikasi	&	Halaman \textit{login} akan ditampilkan	&	sesuai	\\ \hline
				2.	&	Pengguna memasukkan \textit{email} dan \textit{password}	&	Jika \textit{email} dan \textit{password}	sesuai, pengguna akan diarahkan ke halaman utama. & sesuai\\ \hline
				3.	&	Pengguna memilih menu ``Perwalian'' &	Jika pengguna belum memiliki riwayat nilai(masih menempuh semester 1), akan ditampilkan pesan ``PRASYARAT BELUM TERSEDIA''	&	sesuai	\\ \hline
					&	&	Jika pengguna sudah memiliki riwayat nilai	akan ditampilkan tabel prasyarat mata kuliah beserta status pengambilannya dan ringkasan data akademik mahasiswa berupa IPS semester terakhir, IPK, IP Lulus, IP N. Terbaik, SKS Lulus, dan nilai TOEFL &	sesuai	\\ \hline
				4.	&	Pengguna memilih menu ``Jadwal Kuliah'' &	Jika pengguna belum melakukan FRS, cuti studi, atau jadwal kuliah pengguna belum tersedia, akan ditampilkan pesan ``JADWAL KULIAH BELUM TERSEDIA''	&	sesuai	\\ \hline
					&	&	Jika jadwal kuliah pengguna sudah tersedia, akan ditampilkan jadwal kuliah dalam bentuk kalendar yang sudah diurutkan berdasarkan hari &	sesuai	\\ \hline
				5.	&	Pengguna memilih menu ``Syarat Kelulusan'' &	Jika pengguna belum memiliki riwayat nilai(masih menempuh semester 1), akan ditampilkan seluruh mata kuliah wajib dan sks lulus adalah 0 &	sesuai	\\ \hline
					&	&	Jika pengguna sudah memiliki riwayat nilai, akan ditampilkan syarat kelulusan yang belum dipenuhi dari mahasiswa &	sesuai	\\ \hline
				6.	&	Pengguna memilih tombol \textit{logout}	&	Pengguna akan diarahkan kembali ke halaman \textit{login} &	sesuai	\\ \hline
				7.	& Dua pengguna menggunakan aplikasi secara bersamaan	&	Pengguna dapat menggunakan aplikasi dengan akun yang sesuai &	sesuai	\\ \hline
				\end{tabular}
				\label{table:hasilFungsional}
			\end{table}
			
\subsection{Pengujian Eksperimental}
Pengujian eksperimental dilakukan terhadap mahasiswa angkata 2013 sampai 2017. Dari setiap angkatan, diambil satu orang untuk melakukan pengujian. Setiap responden diminta untuk melakukan \textit{login} kemudian melihat dari setiap halaman pada Student Portal dan memastikan apakah data tersebut sesuai dengan data sebenarnya.

Hasil pengujian eksperimental lainnya dirangkum sebagai berikut:
\begin{itemize}
	\item Angakatan 2013
	\item Angkatan 2014
	\item Angkatan 2015
	\item Angkatan 2016
	\item Angkatan 2017
\end{itemize}